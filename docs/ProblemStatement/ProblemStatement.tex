\documentclass{article}

\usepackage{tabularx}
\usepackage{booktabs}
\usepackage{xspace}
\usepackage{url}
\title{CAS 741: Problem Statement\\Translator: from description to C++ and \LaTeX\xspace code}

\author{Bo Cao, caob13@mcmaster.ca}

\date{\today}

\input{../Comments}
\usepackage{color}
\begin{document}

\maketitle
\section{Goal}
The goal is to translate a set of user-written differential-algebraic equations (DAEs) to the corresponding C++ function and \LaTeX\xspace code. The C++ function has the signature required by DAETS, a DAE solver\footnote{\url{ http://www.cas.mcmaster.ca/~nedialk/daets/}}, and it mainly computes the left-hand sides of the DAEs hard-coded inside itself. The \LaTeX\xspace code is an aligned equation environment containing all the equations.

\section{Motivation}
This part of motivation mainly comes from the functionality of some literal programming languages. These languages only requires the users to write mathematical expressions, and the corresponding computing codes are automatically generated. Compared to these languages, our program is tailored to the DAETS solver and the aligned equations environment, so it is more compatible.

\section{Environment}
This program is mainly designed to work under Ubuntu 18.04.3. Compatibility with other environments is possible, but not guaranteed nor tested.

\section{Stakeholders}
The major stakeholders are the users of DAETS, since this program is tailored for DAETS. Other people might be interested as well, if they can utilize our generated C++ functions and/or \LaTeX\xspace codes in other ways.
 
%\begin{table}[hp]
%\caption{Revision History} \label{TblRevisionHistory}
%\begin{tabularx}{\textwidth}{llX}
%\toprule
%\textbf{Date} & \textbf{Developer(s)} & \textbf{Change}\\
%\midrule
%Date1 & Name(s) & Description of changes\\
%Date2 & Name(s) & Description of changes\\
%... & ... & ...\\
%\bottomrule
%\end{tabularx}
%\end{table}
%
%Put your problem statement here.  Comments to you can be added, like this:
%
%\wss{comment}
%
%You can also leave comments for yourself, like this:
%
%\an{comment}

\end{document}